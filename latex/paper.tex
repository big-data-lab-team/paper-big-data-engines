\documentclass[conference]{IEEEtran}
\IEEEoverridecommandlockouts
% The preceding line is only needed to identify funding in the first footnote. If that is unneeded, please comment it out.
\usepackage{cite}
\usepackage{amsmath,amssymb,amsfonts}
\usepackage{graphicx}
\usepackage{textcomp}
\usepackage{enumitem}
\usepackage{xcolor}
\usepackage{times}
\usepackage[binary-units=true]{siunitx}
\usepackage{latexsym}
\usepackage{hyperref}
\hypersetup{colorlinks=true,linkcolor=black,citecolor=blue,filecolor=black,urlcolor=blue}
\usepackage{algpseudocode}
\usepackage{algorithm}
\usepackage{caption}
\usepackage{subcaption}

%\setlength{\textfloatsep}{7pt}

\newcommand{\TG}[1]{\color{cyan}From Tristan: #1 \color{black}}
\newcommand{\MD}[1]{\color{magenta}From Mathieu: #1 \color{black}}
\newcommand{\VHS}[1]{\color{green}From Valerie: #1 \color{black}}

\algnewcommand\algorithmicforeach{\textbf{for each}}
\algdef{S}[FOR]{ForEach}[1]{\algorithmicforeach\ #1\ \algorithmicdo}
\def\BibTeX{{\rm B\kern-.05em{\sc i\kern-.025em b}\kern-.08em
  T\kern-.1667em\lower.7ex\hbox{E}\kern-.125emX}}

\begin{document}

\title{Impact of engines scheduler on the performance of Big Data pipelines in HPC \TG{Suggested title: Performance comparison of Dask and Apache Spark on HPC systems}}

\author{Mathieu Dugr\'e, Val\'erie Hayot-Sasson, Tristan Glatard\\
	Department of Computer Science and Software Engineering\\
	Concordia University, Montr\'eal, Qu\'ebec, Canada\\
	\{mathieu.dugre, valerie.hayot-sasson, tristan.glatard\}@concordia.ca
	\vspace*{0.8cm} % to avoid weird spacing of 1st page by Latex.
}

\maketitle

\begin{abstract}
	% TODO
\end{abstract}

\begin{IEEEkeywords}
	Performance, Scheduler, Big Data, Dask, Spark, Neuroimaging
\end{IEEEkeywords}
\MD{Discuss which keyword to use for the paper.}

\section{Introduction}

\section{Background}
\subsection{Lustre} % TODO

\subsection{Dask}
Dask is a Python-based Big Data engine with growing popularity in the scientific Python ecosystem.
Dask was designed with data locality and in-memory computing in mind, to mitigate the data transfer bottleneck in Big Data workflows.
Data locality, popularized by Map-Reduce \cite{dean2008mapreduce}, schedules tasks where the data reside.
In-memory computing minimizes the overhead of transferring data to disk by keeping data in memory when possible.
Dask uses lazy evaluation to reduce unnecessary communication and computation.
The engine builds a dynamic graph before execution, allowing it to determine which task to compute.
Dask workflows can further reduce data transfer by leveraging multithreading whenever Python's GIL does not restrict it.
Fault-tolerance is achieved by recording data lineage: the sequence of operations used to modify the initial data.

Dask offers five data structures:
\href{https://docs.dask.org/en/latest/array.html}{Array},
\href{https://docs.dask.org/en/latest/bag.html}{Bag},
\href{https://docs.dask.org/en/latest/dataframe.html}{DataFrame},
\href{https://docs.dask.org/en/latest/delayed.html}{Delayed},
and \href{https://docs.dask.org/en/latest/futures.html}{Futures}.
Arrays offer a clone of NumPy API for distributed processing of large arrays.
Bags are a distributed collection of Python object that offers a programming abstraction similar to \href{https://toolz.readthedocs.io/en/latest/}{PyToolz}.
Dataframes are a parallel composition of \href{https://pandas.pydata.org/}{Pandas} Dataframes used to process a large amount of tabular data.
Dask Delayed offers an API for distributing arbitrary functions that do not fit in the above frameworks.
Lastly, Dask Futures can also execute arbitrary functions; however, it launches computation immediately rather than lazily.
Dask modularity allows users to install only required components making it lightweight.

In Dask, a scheduler decides where and when to execute tasks using the Dask graph.
API operations generate multiple fine-coarse tasks in the computation graph, allowing a more straightforward representation of complex algorithms.

The Dask engine is compatible with multiple distributed schedulers, including YARN and Mesos.
Dask also provides its own \textit{Dask Distributed scheduler}.
We chose to use Dask Distributed scheduler to keep the environment balanced between the engines.

In the Dask Distributed scheduler, a \textit{dask-scheduler} process administrates the resource provided by  \textit{dask-worker}s in the cluster.
The scheduler receives jobs from clients and assigns tasks to available workers.
Task scheduler uses a FIFO (First-In-First-Out) job scheduling policy.
That is an utter process branch of the Dask graph before proceeding with the next one.

Dask offers multiple ways to deploy a cluster, including, but not limited to, SSH configs, Kubernetes, SLURM, PBS.
For our experiments, we used the \href{https://jobqueue.dask.org/en/latest/generated/dask_jobqueue.SLURMCluster.html}{Dask SLURM cluster} API.

Dask \MD{Add version when after locking it for experiments.} was used. \TG{should be in methods}

\subsection{Apache Spark}
Apache Spark is a widely-used general-purpose Big Data engine.
Like Dask, it aims at reducing data transfer costs by incorporating data locality, in-memory computing, and lazy evaluation.

Spark offers three options to schedule jobs: Spark Standalone, Mesos, and YARN.
Spark Standalone is a simple built-in scheduler.
YARN is mainly used to schedule Hadoop-based workflows, while Mesos can be used for various workflows.
We limit our focus to Spark Standalone scheduler, as researchers are likely to execute their workflows in an HPC environment where, usually, neither YARN nor Mesos is available.

In the Spark Standalone scheduler, a \textit{leader} \TG{I am all in favor of inclusive terminology, but if Spark didn't update their wording we shouldn't do it for them} coordinates the resource provisioned by \textit{workers} in the cluster.
A \textit{driver}  process receives jobs from clients and requests workers from the leader.
Jobs are divided into stages to be executed onto workers.
Each operation in a stage is represented by a high-level task in the computation graph.
Like Dask, Spark Standalone scheduler uses a FIFO policy to schedule tasks.
Spark Standalone has two execution modes: (1) the client mode, where the driver process runs in a dedicated process,
and (2) the cluster mode, where the driver runs within a worker process.
Our experiments use the client mode since cluster mode is not available in PySpark.

Spark's primary data structure is Resilient Distributed Dataset (RDD)\cite{RDD}, a fault-tolerant, parallel collection of data elements.
RDDs are the basis of the other Spark data structure: Datasets and DataFrames.
Datasets are similar to RDD but benefit additional performance by leveraging the Spark SQL's optimized execution engine. 
The DataFrames are Datasets organized into named-columns and are used to process tabular data. 
While the DataFrame API is available in all supported languages, Datasets are limited to Scala and Java. 

Python is a standard programming language in the scientific community, offering numerous data processing libraries.
While serialization from Python to Java, an operation required when using Spark's Python API, creates overhead, we found it minimal \cite{8943502}.
We focus on PySpark API to have a more balanced environment between the different engines and for its suitability to neuroimaging.

\section{Methods}
\subsection{Infrastructure}
For our experiments we used the SlashBin cluster at Concordia University.
Each compute node has 2 16-cores CPUs,
\SI{256}{\giga\byte} memory, \MD{find out what's the memory speed and \#-channel supported.}
\SI{2.88}{\tera\byte} SSDs of mounted storage.
The nodes are interconnected with a dedicated \SI{10}{\giga\bit/\second} Ethernet connection.
Centos \MD{Verify version} and Linux kernel \MD{Verify kernel version} is installed on the nodes.

\MD{Write about the worker/leader setup once the dataset size is determined. Need to tune so that 2-3 blocks can fit in the excutors memmory}


\subsection{Dataset}
We used BigBrain\cite{Amunts:13}, a 3-D image of the human brain with voxel intensity ranging between 0 and 65,535.
We converted the blocks into the NIfTI format, a popular format in neuroimaging.
We left the NIfTI blocks uncompressed, resulting in a total data size of \SI{00}{\giga\byte}\MD{Verify the size of the dataset when Lustre back on}.
To evaluate the effect of block size, we resplit these blocks into 30, 125 and 750 blocks of \MD{size for each block segment}, using the \href{https://github.com/big-data-lab-team/sam}{sam} library.
	
We also used the dataset provided by the Consortium for Reliability and Reproducibility (\href{http://fcon_1000.projects.nitrc.org/indi/CoRR/html/}{CoRR}) \cite{zuo2014open}.
This dataset is freely available on \href{https://datasets.datalad.org/?dir=/corr/RawDataBIDS}{datalad}.
The entire dataset is \SI{408.4}{\giga\byte}, containing anatomical, diffusion and functional images of 1,397 subjects acquiredin 29 sites.
We used all 3,491 anatomical images, representing \SI{39}{\giga\byte} overall (\SI{11.17}{\mega\byte} per image on average).
	
\subsection{Applications}
\subsubsection{Increment}
We adapted the increment application used in \cite{hayot2019performance}.
This synthetic application reads blocks of the BigBrain from Lustre and simulates computation by sleeping for a specified period.
To simulate intermediate results, we repeat the sleep process for a configurable amount of time.
We prevent data caching of the blocks by incrementing their voxels value by one after each sleep operation.
Finally, we write the resulting NIfTI image back to Lustre.
This application allows us to study the engines when their inputs are processed independently.
The map-only scenario of this application mimics the processing of multiple independent subjects in parallel.
\MD{TODO: Add the Figure for the application graph.}
	
\subsubsection{Multi-Increment}
Our second application is an adaptation of the increment application.
A significant difference is that, at each iteration, it uses a random BigBrain block as the increment value. 
This change allows the multi-increment application to have inter-worker communication while remaining simple.
\MD{TODO: Add the Figure for the application graph.}
	
\subsubsection{Histogram}
As our third application, we calculate the histogram of the BigBrain image. 
The application reads the BigBrain blocks from Lustre, calculates each intensity's frequency, and then writes the aggregated result back on Lustre.
This map-reduce application has a very high read overwrite ratio.
Moreover, this application requires shuffling, albeit of a limited amoutn of data. 
The amount of inter-worker communication is in-between the increment and multi-increment applications.
\MD{TODO: Add the Figure for the application graph.}
	
\subsubsection{Kmeans} % TODO
For our fourth application, we apply Kmeans clustering to the voxel intensities of the BigBrain image.
We set the number of clusters to 3, to segment the white and grey matter and the noise.
The application starts by reading the image blocks, combining all voxels in a 1-D array, and choosing initial centroids using the min, max, and intermediate values.
It assigns each voxel to the centroids it is the closest and updates each centroid by computing the average of the voxels associated with it. 
It repeats the assignment and update steps for a configurable amount of time. 
Finally, the voxels of the image blocks are classified and written back to the file system.
Updating the centroids involves substantial data communication between the workers.

For this application, the Spark and Dask implementations differ slightly,  to take advantage of the best-suited API from both engines.
The Spark implementation uses the Map-Reduce paradigm, while the Dask one uses array programming.
\MD{TODO: Add the Figure for the application graph.}
	
	
\subsubsection{BIDS app example}
Our last application is BIDS App example: a neuroimaging pipeline to measure the brain volume from MRIs.
For this application, we use the CoRR dataset.
The application extracts the brain volume of each participant, then computes the average for each group of participants.
Unlike the other applications, BIDS App example is a command-line executed in a Docker image (bids/example on DockerHub).
We converted the Docker image to a Singularity image for use in HPC environments, \MD{Cite paper on reason why this is done.}
using \href{https://hub.docker.com/r/singularityware/docker2singularity/tags/}{docker2singularity}
\MD{Need to discuss if we preload the image on the worker nodes.}
\MD{TODO: Add the Figure for the application graph.}
	
\subsection{Experiments} % TODO
	
\section{Results} % TODO
	
\section{Discussion} % TODO
	
\section{Conclusion} % TODO
	
\section*{Acknowledgment} % TODO
	
	
\bibliographystyle{IEEEtran}
\bibliography{IEEEabrv,reference}
	
\end{document}
